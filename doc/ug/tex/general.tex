%%% description of general aspects of the package

The top level package resides below {\tt pni.io}. 

\section{Exceptions}

\subsection{{\tt pni.io.io\_error}}

A general exception thrown when data IO fails. It should be used in case of
hardware or system failures originating not from the library or user code but
from somewhere within the OS or the hardware.

\subsection{{\tt pni.io.link\_error}}

In general this exception is associated to two situations. On would be that
the linking of Nexus objects fails. The other one would be errors related to
links on file system level.

\subsection{{\tt pni.io.parser\_error}}

This exception indicates that an ASCII or binary parser failed for some reason.
The reasons for such an exception can be manifold: a malformed  input string, an
error in the parser itself, etc. 

\subsection{{\tt pni.io.invalid\_object\_error}}

This is typically thrown in the user tries to request any kind of data from an
object which is not properly initialized. Consider here the case of objects
which can be default constructed but need further initialization at runtime. 
If the user tries to early to request data from such an object this kind of
exception will be thrown. 

\subsection{{\tt pni.io.object\_error}}

A very general exception which is thrown in all other cases. It typically
indicates that the construction (either during the call of a constructor or
during copying an object) has failed.

\todo[inline]{Still need to move the exceptions to this package}
