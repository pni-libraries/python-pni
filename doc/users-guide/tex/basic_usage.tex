%%%basic usage of the python bindings for libpninx

In this chapter the basic useage of the pninx python bindings will be discussed.
The code snippets shown in this chapter are taken from the examples directory of
the \pninx-Python directory.

\nxgroup and \nxfield\ objects have a couple of read only attributes that can be used to obtain
basic information about an object. For the next considerations it is not of
importance whether we deal with a field or a group object.
Let us assume that  {\tt object} can be addressed with the path 
{\tt /scan\_1/detector/data}. The meaning of the three attributes 
{\tt name}, {\tt base}, and {\tt path} of {\tt object} should be obvious from
the next short code snippet:
\begin{verbatim}
>> print object.name
data
>> print object.base
/scan_1/detector
>> print object.path
/scan_1/detector/data
\end{verbatim}

%%%============================================================================
\section{Handling files}
To create a new or open an existing nexus file the Python package provides two
factory functions {\tt create\_file} and {\tt open\_file}. For both functions
the first positional argument is mandatory. It is a string with the name (path)
of the file top create or open.
\inputminted[linenos=true]{python}{../examples/nxfile_ex1.py}
{\tt create\_group} can take also an optional keyword argument name {\tt class}
which describes the Nexus class the new group should refer too. 
In addition to the filename {\tt create\_file} takes two optional keyword
arguments:
\begin{description}
 \item[{\tt overwrite}] a boolean value. If {\tt True} an already existing 
 file with the same name will be truncated (its content will be deleted). 
 If {\tt False} an exception will be thrown.
 \item[{\tt splitsize}] NOT IMPLEMENTED YET.
\end{description}
{\tt open\_file} has only one additional keyword argument: {\tt readonly}.
If set to {\tt True} the file will be opened in read-only mode.

The Nexus standard defines a couple of attributes that must be attached to the
root group of a file. You do not have to take care about these attributes. They
are managed automatically by the library. 

%%%============================================================================
\section{Groups and attributes}
Groups can be created by instances of \nxgroup\ or \nxfile\ using their 
{\tt create\_group} method.
There are several possibilities how the name of a new group can influence where
the group will be created. For the next examples we assume that {\tt g} is an
instance of \nxgroup\ with path {\tt /scan\_1/instrument}. 
Consider now the following code snippet
\begin{verbatim}
g.create_group("tth")
\end{verbatim}
which will create a new group with path {\tt /scan\_1/instrument/tth}.
Intermediate groups will be created automatically. So for instance 
\begin{verbatim}
g.create_group("monochromator/positioner")
\end{verbatim}
will create a new group with path {\tt
/scan\_1/instrument/monochromator/positioner}. The group {\tt monochromator} 
does not have to exist at the time the above line of code is called. 
By using an absolute path for the name the new group will be created wherever
the path of the new group points to. 
\begin{verbatim}
g.create_group("/scan_1/sample/omega")
\end{verbatim}
{\tt create\_group} takes also an optional keyword argument {\tt nxclass} which
describes the Nexus object the newly created group shall belong too. This is
typically a Nexus base class. See the Nexus reference manual for more
information on base classes.

The following example shows how to create groups using instances of \nxgroup\
and \nxfile.
\inputminted[linenos=true]{python}{../examples/nxgroup_ex1.py}
Lines 8-11 show the creation of new groups. To open an existing group one can
either use the {\tt open} method belonging to each instance of \nxgroup\ and 
\nxfile. Alternatively one can use the {\tt []} operator along with the path of
the object one wants to open. Lines 14, 15, and 18 show code where existing
groups are opened.

\section{Handling attributes}

To each object of type \nxgroup, \nxfile, and \nxfield\ attributes can be
attached. Attributes are objects of type \nxattribute. 
Objects of type \nxattribute\ are extremely simple things. All IO as well as
information about an attribute object can be done or obtained via the attributes
of such an object. 
Each attribute has three read only attributes name {\tt name}, {\tt dtype}, and
{\tt shape}. The first one returns the name of the attribute. The second one the
type of the attribute as numpy typecode. The last attribute {\tt shape}
represents the shape of an attribute as a tuple.
Reading and writing data to an attribute can be done via {\tt value} attribute.

To create an attribute use the {\tt attr} method provided by all objects. 
This method requires to positional arguments: the name of the attribute and its 
typecode. To create for instance a string attribute with name {\tt text} use
\begin{verbatim}
a = obj.attr("text","string")
\end{verbatim}
The additional keyword argument {\tt shape} can be used to create
multidimensional attributes
\begin{verbatim}
a = obj.attr("cij","float32",shape=(6,6))
\end{verbatim}
where {\tt shape} is a tuple determining the number of dimensions as well as the
number of elements along each dimension.
Data can be written or read from an attribute using the {\tt value} member
attribute of the attribute object
\begin{verbatim}
a.value = "hello world"
\end{verbatim}
If the dimensionality or data type does not match an exception will be thrown. 


In the next example we will have a look how to handle attributes.
\inputminted[linenos=true]{python}{../examples/nxgroup_ex2.py}
\inputminted[linenos=true]{python}{../examples/nxgroup_ex3.py}

%%%============================================================================
\section{Creating fields}

%%%============================================================================
\section{A simple example}
\inputminted[linenos=true]{python}{../examples/simple_io.py}
